\documentclass[a4paper]{article}
\usepackage{hyperref}
\begin{document}
\title{Pylux User Manual}
\author{J. Page}
\date{2015}
\maketitle
\tableofcontents
Copyright (C)  2015 Jack Page \\
Permission is granted to copy, distribute and/or modify this document
under the terms of the GNU Free Documentation License, Version 1.3
or any later version published by the Free Software Foundation;
with no Invariant Sections, no Front-Cover Texts, and no Back-Cover Texts.
A copy of the license is included in the section entitled "GNU
Free Documentation License".
\newpage
\section{Introduction}
Pylux is a program written in Python for the manipulation of OpenLighting Plot
documents. Pylux currently allows for the basic editing of the OpenLighting
Plot XML files, including referencing OpenLighting Fixture files to obtain
additional data.

In the future, Pylux will be extended with modules that allow for the exporting
of documentation in the \LaTeX{} format, and creating lighting plots in SVG 
format.

Bugs and feature requests should be submitted to 
\url{https://github.com/jackdpage/pylux/issues}.

\section{Command-line Options}
Pylux is invoked simply by running the \texttt{plotter.py} program. Windows 
operating systems should automatically associate file with the Python 
interpreter. The program contains a shebang line for compatibility with UNIX
systems. 

\paragraph{\texttt{--help}, \texttt{-h}}
Display the usage message of the program then exit.

\paragraph{\texttt{--version}, \texttt{-v}}
Display the version number of the program then exit.

\paragraph{\texttt{--file \textit{FILE}}, \texttt{-f \textit{FILE}}} 
Load the file with path \texttt{\textit{FILE}} as the project file on launch.

\paragraph{\texttt{--interface \textit{INTERFACE}}, \texttt{-i 
\textit{INTERFACE}}}
Launch Pylux with the interface with identifier \texttt{\textit{INTERFACE}}.
\texttt{\textit{INTERFACE}} is either \texttt{cli} or \texttt{gui} depending on 
whether you want to launch the command line interface or graphical user 
interface respectively. The default is \texttt{cli}.

\section{Basic Concepts}
Pylux uses the OpenLighting Plot format as its document format. This format 
specifies three main components of an XML document: the meta section, the 
fixtures section and the registries section. In its current version, Pylux 
can only read and write to the latter two sections.

The fixtures section contains multiple fixture elements, each of which has 
various data associated with it, depending on the fields as defined by the 
fixture's OpenLighting ID (olid).

The registries section contains one or more DMX registries, each of which 
contains some channel elements, which contain both the UUID of the fixture 
they control and their function.

\section{Using the Command Line Interface}
Pylux ships with two interface modes: a command line interface (CLI) and a 
graphical user interface (GUI). The interface that is loaded on launch is 
specified by the interface flag when the program is run. However, the 
default interface is the CLI. When the CLI launches, you are presented with an 
interactive prompt, which accepts commands which may also have arguments. Each 
command is two letters long (apart from the utility commands), where the first 
letter represents the aspect that the command affects and the second letter 
represents the action of the command.

\subsection{Utility Commands}

\paragraph{\texttt{h}}
Display a list of the available commands for the interactive prompt. This 
prints the contents of \texttt{help.txt}  to the output.

\paragraph{\texttt{c}}
Clears the screen of previous input and output. This uses the system screen 
clearing command. (\texttt{clear} on UNIX, \texttt{cls} on Windows)

\paragraph{\texttt{q}}
Exit the program and autosave any changes that have been made to the tree.

\paragraph{\texttt{q!}}
Exit the program without saving any changes to disk.

\subsection{File Commands}

\paragraph{\texttt{fl \textit{FILE}}}
Load the file with path \texttt{\textit{FILE}} as the project file. This will 
override any unsaved buffer associated with the previous project file, if there 
was one.

\paragraph{\texttt{fs}}
Save the current file buffer to its original location.

\paragraph{\texttt{fS \textit{PATH}}}
Save the current file buffer to a new file with location \texttt{\textit{PATH}}.

\subsection{Metadata Commands}

\paragraph{\texttt{ml}}
List all the metadata tags associated with the currently loaded project
file.

\paragraph{\texttt{ma \textit{TAG VALUE}}}
Add a new piece of metadata. Adds a new tag with the name \texttt{\textit{TAG}}
to the metadata section and sets its value to \texttt{\textit{VALUE}}.

\paragraph{\texttt{mr \textit{TAG}}}
Remove the piece of metadata from the file which has the name 
\texttt{\textit{TAG}}.

\paragraph{\texttt{mg \textit{TAG}}}
Get the value of a piece of metadata. Prints the value of the metadata with
name \texttt{\textit{TAG}} on the screen.

\subsection{Fixture Commands}

\paragraph{\texttt{xa \textit{OLID}}}
Add a new fixture to the plot. This will add a fixture to the plot with olid 
\texttt{\textit{OLID}}. It will also automatically assign a UUID and search the 
OLF file associated with this olid for any constants that this fixture has and 
add them to the fixture. This will not assign DMX addresses to the fixture,
use \texttt{xA} for that.

\paragraph{\texttt{xl}}
List all the fixtures in the plot. This will generate a list of every fixture 
in the plot, listing an interface reference, the fixture olid, and the fixture 
UUID.

\paragraph{\texttt{xf \textit{TAG VALUE}}}
List all the fixtures in the plot which have a tag called \texttt{\textit{TAG}} 
with a value of \texttt{\textit{VALUE}}. Like the list function, this will list 
an interface reference, the fixture olid and UUID, and also the value of the 
tag that was given for verification purposes.

\paragraph{\texttt{xi \textit{REF}}}
List all the information associated with the fixture with interface reference 
\texttt{\textit{REF}}. \texttt{\textit{REF}} is the number given to a fixture 
by either the list or filter commmand. It must be the number given by the most 
recently run command, as this interface reference buffer is updated each time 
one of these commands is run.

\paragraph{\texttt{xr \textit{REF}}}
Remove the fixture with the interface reference \texttt{\textit{REF}}. This 
fixture will be removed from the plot, but associated DMX channels will not be 
removed, use \texttt{xp} for that.

\paragraph{\texttt{xs \textit{REF TAG VALUE}}}
Set the value of \texttt{\textit{TAG}} in fixture with interface reference 
\texttt{\textit{REF}} to \texttt{\textit{VALUE}}.

\paragraph{\texttt{xA \textit{REF UNIVERSE ADDR}}}
Assign DMX addresses to the fixture with interface reference 
\texttt{\textit{REF}}. This will add the fixture to the universe 
\texttt{\textit{UNIVERSE}}, beginning at the start address 
\texttt{\textit{ADDR}}. \texttt{\textit{ADDR}} can either be a user-assigned 
number or auto to allow Pylux to choose the most appropriate start address.

\subsection{DMX Registry Commands}

\paragraph{\texttt{rl \textit{UNIVERSE}}}
List all the used channels in \texttt{\textit{UNIVERSE}}. This will list the 
DMX address, fixture UUID and function of every channel in the DMX registry 
with identifier \texttt{\textit{UNIVERSE}}.

\section{Using the Graphical User Interface}
You may instead choose to launch Pylux using its GUI. This is NYI so please 
don't.

\section{Extensions}
In the previous sections, we have discussed the usage of the base program in 
Pylux: \texttt{plotter}. This is the program that you will use to edit your
plots, however, beyond that, it doesn't do much. That is why Pylux can be 
extended with additional Python scripts to provide extra functionality. In 
the current version, Pylux comes bundled with the \texttt{genlux} script only.

\subsection{TeXlux}
TeXlux is an extension to the base \texttt{plotter} program which allows for 
the creation of reports in the \LaTeX{} format, which can then be 
post-processed to create a PDF file, or many other formats through the use of 
an external tool such as Pandoc.

\subsubsection{Invoking}
TeXlux is invoked as a separate program, although it does require 
\texttt{plotter} to be present as it is imported when TeXlux is run. TeXlux is 
invoked using: \texttt{texlux \textit{FILE} \textit{TEMPLATE}}, where 
\texttt{\textit{FILE}} is the Pylux plot file to process and 
\texttt{\textit{TEMPLATE}} is the built-in function to create the report from.

\subsubsection{Command-line Options}
TeXlux has very little in the way of command-line options.

\paragraph{\texttt{--title \textit{TITLE}}, \texttt{-t \textit{TITLE}}}
Sets the title of the report to be \texttt{\textit{TITLE}}. If this is not 
specified, a generic title such as 'Dimmer Report' will be used.

\subsubsection{Processing}

TeXlux uses built-in functions to generate \LaTeX{} docments with pre-defined 
structures. Each built-in function has a corresponding \LaTeX{} style file 
installed in \texttt{~/.pylux/tex/} which is required to build the PDF report. 
Currently the only built-in function is \texttt{dimmer}, which produces a 
report categorised by dimmer and containing power draw totals.

\subsubsection{Output}

TeXlux generates the \LaTeX{} source in STDOUT, which can be written to a 
file or piped directly into a \LaTeX{} processing tool such as 
\texttt{pdflatex}. Any errors that were encountered during processing are 
written as \LaTeX{} compatible comments at the top of the source.

\section{GNU Free Documentation License}
\input{fdl}

\end{document}
