\documentclass[a4paper]{article}
\usepackage{hyperref}
\begin{document}
\title{Pylux User Manual \\ \large{for Pylux v0.1-alpha2}}
\author{J. Page}
\date{2016}
\maketitle
\vspace{1.0pt}
Copyright (C)  2015 2016 Jack Page \\
Permission is granted to copy, distribute and/or modify this document
in source form (LaTeX) or compiled form (PDF, PostScript, etc.), including 
for commercial use, provided that this copyright notice is retained and that 
you grant the same freedoms to any recipients of your modifications.
\tableofcontents
\newpage
% ------- END HEADER MATERIAL ------- %
\section{Introduction}
\subsection{About Pylux}
Pylux is a program written in Python for the manipulation of OpenLighting Plot
documents. Pylux currently allows for the basic editing of the OpenLighting
Plot XML files, including referencing OpenLighting Fixture files to obtain
additional data.

In the future, Pylux will be extended with modules that allow for the exporting
of documentation in the \LaTeX{} format, and creating lighting plots in SVG 
format.

Bugs and feature requests should be submitted to 
\url{https://github.com/jackdpage/pylux/issues}.

\subsection{About this Manual}
This manual is intended for general users of Pylux. If you are a developer 
who wants to contribute to Pylux, you should read the Developer Reference. 
The first section of this manual details all of the actions that you can enter 
on the command-line interface of Pylux. The second section goes into more 
detail about what tags you should use, etc.

% ------- USER SECTION ------- %
\section{Command-line Options}
Pylux is invoked simply by running the \texttt{pylux} package with Python. 

\paragraph{\texttt{--help}, \texttt{-h}}
Display the usage message of the program then exit.

\paragraph{\texttt{--version}, \texttt{-v}}
Display the version number of the program then exit.

\paragraph{\texttt{--file \textit{FILE}}, \texttt{-f \textit{FILE}}} 
Load the file with path \texttt{\textit{FILE}} as the plot file on launch.

\paragraph{\texttt{--gui}, \texttt{-g}}
Launch Pylux with the graphical user interface. This is not yet implemented. 
Omitting this flag launches Pylux with its standard CLI interface.


\section{Using the Command Line Interface}
Pylux ships with two interface modes: a command line interface (CLI) and a 
graphical user interface (GUI). The interface that is loaded on launch is 
specified by the interface flag when the program is run. However, the 
default interface is the CLI. When the CLI launches, you are presented with an 
interactive prompt, which accepts commands which may also have arguments. Each 
command is two letters long (apart from the utility commands), where the first 
letter represents the aspect that the command affects and the second letter 
represents the action of the command.

When you are using the CLI, you will notice that some commands call for you 
to provide an interface reference id (usually called \texttt{\textit{REF}} 
in the command synopsis). This allows you to easily pass objects that have 
been listed on-screen into another command. You can tell which references are 
legal because they will be underlined. For example to pipe a fixture object 
into the \texttt{xs} command, you would first run a command which lists the 
fixture you wish to change (\texttt{xl} or \texttt{xf}), find the interface 
reference of that fixture, then use that reference in the \texttt{xs} 
command. In addition to these references, there is an additional special 
reference which is not displayed on-screen which is called using the 
reference \texttt{this}. This special reference will refer to the object that 
was previously worked on. For example, if you perform \texttt{xs} on a fixture 
then wish to perform another \texttt{xs} immediately afterwards, you can use 
\texttt{this} instead of the interface reference.

To allow for the easy editing of objects in batch, you can specify more than 
one interface reference at once using a comma separated list such as a,b,c. 
You can also specify ranges such as a:b and combine these two features such 
as a:b,c,d,e:f.

\subsection{Utility Commands}

\paragraph{\texttt{h}}
Display a list of the available commands for the interactive prompt. This 
prints the contents of \texttt{help.txt}  to the output.

\paragraph{\texttt{c}}
Clears the screen of previous input and output. This uses the system screen 
clearing command. (\texttt{clear} on UNIX, \texttt{cls} on Windows)

\paragraph{\texttt{q}}
Exit the program and autosave any changes that have been made to the tree.

\paragraph{\texttt{Q}}
Exit the program without saving any changes to disk.

\subsection{File Commands}

\paragraph{\texttt{fo \textit{FILE}}}
Open the file with path \texttt{\textit{FILE}} as the plot file. This will 
override any unsaved buffer associated with the previous plot file, if 
there was one.

\paragraph{\texttt{fw}}
Save the current file buffer to its original location.

\paragraph{\texttt{fW \textit{PATH}}}
Save the current file buffer to a new file with location 
\texttt{\textit{PATH}}.

\paragraph{\texttt{fg}}
Print the path of the file that is currently loaded.

\paragraph{\texttt{fn \textit{PATH}}}
Create an new empty plot file at the location with path \texttt{\textit{PATH}} 
and load it as the new plot file

\subsection{Metadata Commands}

\paragraph{\texttt{mG}}
List all the metadata tags associated with the currently loaded plot file.

\paragraph{\texttt{ms \textit{TAG VALUE}}}
Set the value of the metadata with tag \texttt{\textit{TAG}} to 
\texttt{\textit{VALUE}}. If the metadata already exists, it will be 
overridden.

\paragraph{\texttt{mr \textit{TAG}}}
Remove the piece of metadata from the file which has the name 
\texttt{\textit{TAG}}.

\paragraph{\texttt{mg \textit{TAG}}}
Get the value of a piece of metadata. Prints the value of the metadata with
name \texttt{\textit{TAG}} on the screen.

\subsection{Fixture Commands}

\paragraph{\texttt{xn \textit{TEMPLATE}}}
Add a new fixture to the plot. This will load the contents of the fixture 
file with the name \texttt{TEMPLATE} into the new fixture, including DMX 
functions. This will not allocate DMX addresses to the fixture, use 
\texttt{xA} for that.

\paragraph{\texttt{xc \textit{REF}}}
Clone the fixture with interface reference \texttt{\textit{REF}} into a new 
fixture. This does not reassign any DMX values.

\paragraph{\texttt{xl}}
List all the fixtures in the plot. This will generate a list of every fixture 
in the plot, listing an interface reference, the fixture olid, and the fixture 
UUID.

\paragraph{\texttt{xf \textit{TAG VALUE}}}
List all the fixtures in the plot which have a tag called 
\texttt{\textit{TAG}} with a value of \texttt{\textit{VALUE}}. Like the list 
function, this will list an interface reference, the fixture olid and UUID, 
and also the value of the tag that was given for verification purposes.

\paragraph{\texttt{xg \textit{REF TAG}}}
Print the value of \texttt{\textit{TAG}} for the fixture with interface
reference \texttt{\textit{REF}}.

\paragraph{\texttt{xG \textit{REF}}}
List all the information associated with the fixture with interface reference 
\texttt{\textit{REF}}.

\paragraph{\texttt{xr \textit{REF}}}
Remove the fixture with the interface reference \texttt{\textit{REF}}. This 
fixture will be removed from the plot, but associated DMX channels will not be 
removed, use \texttt{xp} for that.

\paragraph{\texttt{xs \textit{REF TAG VALUE}}}
Set the value of \texttt{\textit{TAG}} in fixture with interface reference 
\texttt{\textit{REF}} to \texttt{\textit{VALUE}}. For a list of standard 
fixture tags, see \autoref{sec:fixtags}. There are also some shortcuts to set 
mulitple tags at once, which can be found in the illegal tags section.

\paragraph{\texttt{xA \textit{REF UNIVERSE ADDR}}}
Assign DMX addresses to the fixture with interface reference 
\texttt{\textit{REF}}. This will add the fixture to the universe 
\texttt{\textit{UNIVERSE}}, beginning at the start address 
\texttt{\textit{ADDR}}. \texttt{\textit{ADDR}} can either be a user-assigned 
number or auto to allow Pylux to choose the most appropriate start address.

\paragraph{\texttt{xp \textit{REF}}}
Remove the fixture with interface reference \texttt{\textit{REF}} from the 
plot and also remove any DMX channels associated with it.

\subsection{DMX Registry Commands}

\paragraph{\texttt{rl \textit{UNIVERSE}}}
List all the used channels in \texttt{\textit{UNIVERSE}}. This will list the 
DMX address, fixture UUID and function of every channel in the DMX registry 
with identifier \texttt{\textit{UNIVERSE}}.

\subsection{Using Extensions}
You cannot directly interact with extensions from the \texttt{plotter} 
interface, you must first load the extension using the \texttt{:} command.
For example, to load the \texttt{texlux} extension, use \texttt{:texlux}. 
This will then present you with the interface as defined by that extension 
which may vary but in practise should be a prompt of the form 
\texttt{pylux:extension} to indicate to you which extension you are 
operating in and some commands, much like in the \texttt{plotter} interface.
The extension defines its own way of returning to \texttt{plotter} but this 
should in general be \texttt{::} or \texttt{q}.

\section{Using the Graphical User Interface}
You may instead choose to launch Pylux using its GUI. This is NYI so please 
don't.

\section{Extensions}
In the previous sections, we have discussed the usage of the base program in 
Pylux: \texttt{plotter}. This is the program that you will use to edit your
plots, however, beyond that, it doesn't do much. That is why Pylux is also 
bundled with extensions to provide extra functionality. In the current 
version, Pylux comes bundled with the \texttt{texlux} extension only.

\subsection{\texttt{texlux}}
\texttt{texlux} is an extension to the base \texttt{plotter} program which 
allows for the creation of reports in the \LaTeX{} format, which can then be 
post-processed to create a PDF file, or many other formats through the use of 
an external tool such as Pandoc.

\subsubsection{Commands}

\paragraph{\texttt{rn \textit{TEMPLATE OUTPUT TITLE}}}
Generates a report using the template \texttt{\textit{TEMPLATE}}, with the 
title \texttt{\textit{TITLE}}, writing the output to a file with path 
\texttt{\textit{OUTPUT}}.

\subsubsection{Processing}

\texttt{texlux} uses built-in functions to generate \LaTeX{} docments with 
pre-defined structures. Each built-in function has a corresponding \LaTeX{} 
style file installed in \texttt{\~{}/texmf} which is required to build 
the PDF report. Currently the only built-in function is \texttt{dimmer}, which 
produces a report categorised by dimmer and containing power draw totals.

\subsection{\texttt{plotgen}}
\texttt{plotgen} generates, from the fixture list and the fixture's symbol 
files, a plan view of the lighting plot in SVG format. Currently WIP but 
functional. Includes support for positioning, rotation and colouring based 
on fixture data.

% -------- FILES SECTION ------- %
\section{Standard Tags} \label{sec:plotfile}
Below is a list of standard tags for each section, to advise which tags you 
should apply to your metadata and fixtures. Also included is a list of 
pseudo-tags: tags which are not added to the file but actually represent one 
or more other tags to make adding them easier.

\subsection{Standard Metadata Tags}
Whilst you can use any name for a tag you wish, there are some standard ones 
which are used by Pylux and its bundled extensions.

\paragraph{\texttt{production}}
The name of the production for which the lighting documentation is being 
produced, e.g. 'Romeo and Juliet'. Used by: \texttt{texlux}.

\paragraph{\texttt{designer}}
The name of the lighting designer for this production. Used by: 
\texttt{texlux}.

\paragraph{\texttt{board\_operator}}
The name of the person operating the main lighting board for this production.
Used by: \texttt{texlux}.

\paragraph{\texttt{spot\_operator}}
The name of the person operating the primary followspot for this production.
Used by: \texttt{texlux}.

\paragraph{\texttt{director}}
The name of the director of the production. Used by: \texttt{texlux}.

\paragraph{\texttt{venue}}
The location at which the production is taking place. Used by: 
\texttt{texlux}.

\subsection{Standard Fixture Data Tags} \label{sec:fixtags}

\paragraph{\texttt{dmx\_functions}}
This is the parent of a list of empty elements, each of which represents a 
function that the fixture has that requires the use of a DMX channel. For 
example, traditional fixtures will have an \texttt{intensity} function 
whilst modern LED fixtures may have \texttt{colour} and \texttt{mode} 
functions. Used by: \texttt{plotter}.

\paragraph{\texttt{dmx\_channels}}
The number of DMX channels that a fixture needs. This is automatically 
calculated from the \texttt{dmx\_functions} tag, so should not be changed 
manually. Used by: \texttt{plotter}.

\paragraph{\texttt{dmx\_start\_address}}
The start address of this fixture, if it has been addressed. This is 
automatically assigned using the address function so shouldn't be changed 
manually. Used by: \texttt{plotter texlux}.

\paragraph{\texttt{universe}}
The universe in which the DMX channels for this fixture are located. This too 
is set when the address command is run so shouldn't be changed by the user. 
Used by: \texttt{plotter}.

\paragraph{\texttt{posX}}
The x position in 2D space where this fixture is located. Measured in metres.

\paragraph{\texttt{posY}}
The y position in 2D space where this fixture is located. Measured in metres.

\paragraph{\texttt{focusX}}
The x position in 2D space where the centre of this fixture's beam is 
focused. Measured in metres.

\paragraph{\texttt{focusY}}
The y position in 2D space where the centre of this fixture's beam is 
focused. Measured in metres.

\paragraph{\texttt{rotation}}
The rotation of this fixture about its centre. Measured anticlockwise from 
the positive x axis in degrees. This can be automatically calculated if the 
preceding four data tags are present.

\paragraph{\texttt{dimmer}}
For traditional fixtures only. The dimmer that is controlling this fixture. 
Used by: \texttt{texlux}.

\paragraph{\texttt{circuit}}
For traditional fixtures only. The circuit into which the fixture is patched. 
Used by: \texttt{texlux}.

\paragraph{\texttt{power}}
For traditional fixtures only. The maximum power draw by the lamp in this 
fixture.

\paragraph{\texttt{gel}}
The name or manufacturer's code of the gel that is being used in this 
fixture.

\paragraph{\texttt{colour}}
The colour in which this fixture should be rendered on lighting plots. This 
can be calculated automatically if the preceding tag is present and is a 
standard colour name or code. (NYI)

\subsubsection{Pseudo Fixture Tags}
These tags can be used to set multiple attributes of a fixture at once.

\paragraph{\texttt{position \textit{X},\textit{Y}}}
Sets \texttt{posX} to \texttt{\textit{X}} and \texttt{posY} to 
\texttt{\textit{Y}}.

\paragraph{\texttt{focus \textit{X},\textit{Y}}}
Sets \texttt{focusX} to \texttt{\textit{X}} and \texttt{focusY} to 
\texttt{\textit{Y}}.

\paragraph{\texttt{dimmer \textit{REF} \textit{CHAN}}}
Sets \texttt{dimmer\_uuid} to the uuid of the dimmer represented by 
\texttt{\textit{REF}} and \texttt{dimmer\_channel} to \texttt{\textit{CHAN}}.

\end{document}
