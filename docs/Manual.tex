\documentclass[a4paper]{article}
\usepackage{hyperref}
\usepackage{longtable}
\begin{document}
\title{Pylux User Manual \\ \large{for Pylux v0.1-alpha2}}
\author{J. Page}
\date{2016}
\maketitle
\vspace{1.0pt}
Copyright (C)  2015 2016 Jack Page \\
Permission is granted to copy, distribute and/or modify this document
in source form (LaTeX) or compiled form (PDF, PostScript, etc.), including 
for commercial use, provided that this copyright notice is retained and that 
you grant the same freedoms to any recipients of your modifications.
\tableofcontents
\newpage
% ------- END HEADER MATERIAL ------- %
\section{Introduction}
\subsection{About Pylux}
Pylux is a program written in Python for the manipulation of OpenLighting Plot
documents. Pylux currently allows for the basic editing of the OpenLighting
Plot XML files, including referencing OpenLighting Fixture files to obtain
additional data.

In the future, Pylux will be extended with modules that allow for the exporting
of documentation in the \LaTeX{} format, and creating lighting plots in SVG 
format.

Bugs and feature requests should be submitted to 
\url{https://github.com/jackdpage/pylux/issues}.

\subsection{About this Manual}
This manual is intended for general users of Pylux. If you are a developer 
who wants to contribute to Pylux, you should read the Developer Reference. 
The first section of this manual details all of the actions that you can enter 
on the command-line interface of Pylux. The second section goes into more 
detail about what tags you should use, etc.

% ------- USER SECTION ------- %
\section{Command-line Options}
Pylux is invoked simply by running the \texttt{pylux} package with Python. 

\paragraph{\texttt{--help}, \texttt{-h}}
Display the usage message of the program then exit.

\paragraph{\texttt{--version}, \texttt{-v}}
Display the version number of the program then exit.

\paragraph{\texttt{--file \textit{FILE}}, \texttt{-f \textit{FILE}}} 
Load the file with path \texttt{\textit{FILE}} as the plot file on launch.

\paragraph{\texttt{--gui}, \texttt{-g}}
Launch Pylux with the graphical user interface. This is not yet implemented. 
Omitting this flag launches Pylux with its standard CLI interface.

\paragraph{\texttt{--verbose}, \texttt{-V}}
Set the verbosity level of output. Don't include to keep logging level at the 
default of \texttt{WARNING}. Include once for logging level of \texttt{INFO} 
and include twice for a logging level of \texttt{DEBUG}.

\section{Using the Command Line Interface}
The command line interface (CLI) is the default interface used by Pylux. It 
allows for very efficient editing of plot files with very little CPU overhead. 

The CLI is interacted with using a series of commands, each of which may have 
one or more arguments. When the prompt is displayed, the program is waiting 
for the user to enter one of these commands. Each command is a memorable 
two character sequence (apart from the utility commands), where the first 
character is the object that is going to be acted upon and the second 
character is the action to perform.

\subsection{Piping Complex Objects into Commands}
Many commands require that you specify an object other than one which can be 
represented by a simple command line string. For example, the \texttt{xs} 
command requires that you supply a fixture object as an argument. This is made 
possible whilst retaining the simplicity of text based entry through the 
interface reference system.

When you run a listing or filtering command such as \texttt{xl}, you will 
notice that the objects in the list each have a number which is underlined. 
This is that objects's interface reference. Using this, you can pipe objects
into other commands, simply by specifying the number where the command calls 
for another object type. 

To allow for the efficient manipulation of objects in batch, you can specify 
more than one object at once using a comma-separated list (provided of course 
that the command allows for multiple objects to be piped into it) such as 
\texttt{a,b,c}. You can also specify ranges of numbers if you are piping 
sequential objects from a list using the format \texttt{a:b} where \texttt{a} 
and \texttt{b} are the inclusive limits. You can of course use any 
combination of these formats, such as \texttt{a:b,c,d:e,f}.

If you need to pipe the same object or group of objects into multiple commands, 
you can use the \texttt{this} reference instead of a number or list of numbers. 
This points to the last used reference, so will pipe in the same object or 
objects that were used for the last command, unless the reference list has 
changed since (i.e. a listing command has been run again).

\subsection{Utility Commands}

\paragraph{\texttt{h}}
Display a list of the available commands for the interactive prompt. This 
prints the contents of \texttt{help.txt}  to the output.

\paragraph{\texttt{c}}
Clears the screen of previous input and output. This uses the system screen 
clearing command. (\texttt{clear} on UNIX, \texttt{cls} on Windows)

\paragraph{\texttt{q}}
Exit the program and autosave any changes that have been made to the tree.

\paragraph{\texttt{Q}}
Exit the program without saving any changes to disk.

\subsection{File Commands}

\paragraph{\texttt{fo \textit{FILE}}}
Open the file with path \texttt{\textit{FILE}} as the plot file. This will 
override any unsaved buffer associated with the previous plot file, if 
there was one.

\paragraph{\texttt{fw}}
Save the current file buffer to its original location.

\paragraph{\texttt{fW \textit{PATH}}}
Save the current file buffer to a new file with location 
\texttt{\textit{PATH}}.

\paragraph{\texttt{fg}}
Print the path of the file that is currently loaded.

\paragraph{\texttt{fn \textit{PATH}}}
Create an new empty plot file at the location with path \texttt{\textit{PATH}} 
and load it as the new plot file

\subsection{Metadata Commands}

\paragraph{\texttt{mG}}
List all the metadata tags associated with the currently loaded plot file.

\paragraph{\texttt{ms \textit{TAG VALUE}}}
Set the value of the metadata with tag \texttt{\textit{TAG}} to 
\texttt{\textit{VALUE}}. If the metadata already exists, it will be 
overridden.

\paragraph{\texttt{mr \textit{TAG}}}
Remove the piece of metadata from the file which has the name 
\texttt{\textit{TAG}}.

\paragraph{\texttt{mg \textit{TAG}}}
Get the value of a piece of metadata. Prints the value of the metadata with
name \texttt{\textit{TAG}} on the screen.

\subsection{Fixture Commands}

\paragraph{\texttt{xn \textit{TEMPLATE}}}
Add a new fixture to the plot. This will load the contents of the fixture 
file with the name \texttt{TEMPLATE} into the new fixture, including DMX 
functions. This will not allocate DMX addresses to the fixture, use 
\texttt{xA} for that.

\paragraph{\texttt{xc \textit{REF}}}
Clone the fixture with interface reference \texttt{\textit{REF}} into a new 
fixture. This does not reassign any DMX values.

\paragraph{\texttt{xl}}
List all the fixtures in the plot. This will generate a list of every fixture 
in the plot, listing an interface reference, the fixture olid, and the fixture 
UUID.

\paragraph{\texttt{xf \textit{TAG VALUE}}}
List all the fixtures in the plot which have a tag called 
\texttt{\textit{TAG}} with a value of \texttt{\textit{VALUE}}. Like the list 
function, this will list an interface reference, the fixture olid and UUID, 
and also the value of the tag that was given for verification purposes.

\paragraph{\texttt{xg \textit{REF TAG}}}
Print the value of \texttt{\textit{TAG}} for the fixture with interface
reference \texttt{\textit{REF}}.

\paragraph{\texttt{xG \textit{REF}}}
List all the information associated with the fixture with interface reference 
\texttt{\textit{REF}}.

\paragraph{\texttt{xr \textit{REF}}}
Remove the fixture with the interface reference \texttt{\textit{REF}}. This 
fixture will be removed from the plot, but associated DMX channels will not be 
removed, use \texttt{xp} for that.

\paragraph{\texttt{xs \textit{REF TAG VALUE}}}
Set the value of \texttt{\textit{TAG}} in fixture with interface reference 
\texttt{\textit{REF}} to \texttt{\textit{VALUE}}. For a list of standard 
fixture tags, see \autoref{sec:fixtags}. There are also some shortcuts to set 
mulitple tags at once, which can be found in the pseudo tags section.

\paragraph{\texttt{xa \textit{REF UNIVERSE ADDR}}}
Assign DMX addresses to the fixture with interface reference 
\texttt{\textit{REF}}. This will add the fixture to the universe 
\texttt{\textit{UNIVERSE}}, beginning at the start address 
\texttt{\textit{ADDR}}. \texttt{\textit{ADDR}} can either be a user-assigned 
number or auto to allow Pylux to choose the most appropriate start address.

\paragraph{\texttt{xp \textit{REF}}}
Remove the fixture with interface reference \texttt{\textit{REF}} from the 
plot and also remove any DMX channels associated with it.

\subsection{DMX Registry Commands}

\paragraph{\texttt{rl \textit{UNIVERSE}}}
List all the used channels in \texttt{\textit{UNIVERSE}}. This will list the 
DMX address, fixture UUID and function of every channel in the DMX registry 
with identifier \texttt{\textit{UNIVERSE}}.

\subsection{Using Extensions}
You cannot directly interact with extensions from the \texttt{editor} 
interface, you must first load the extension using the \texttt{:} command.
For example, to load the \texttt{texlux} extension, use \texttt{:texlux}. 
This will then present you with the interface as defined by that extension 
which may vary but in practise should be a prompt of the form 
\texttt{pylux:extension} to indicate to you which extension you are 
operating in and some commands, much like in the \texttt{editor} interface.
The extension defines its own way of returning to \texttt{editor} but this 
should in general be \texttt{::} or \texttt{q}.

\section{Using the Graphical User Interface}
Pylux also comes with a graphical user interface. This is currently in 
development so most features are not yet implemented. When the GUI is 
launched, you are presented with a window containing a list of the fixtures 
in the plot. (The only way to load a plot file is using the \texttt{-f} tag 
at launch as there is currently no menu.) Each of these fixtures has to the 
right of it three buttons.

The leftmost button is the only currently functional button and launches a 
window containing a list of the fixture's data tags. You cannot currently 
edit the data tags. The other two buttons are for cloning and removing the 
fixture but this functionality is not yet implemented.

\section{Extensions}
In the previous sections, we have discussed the usage of the base program in 
Pylux: \texttt{editor}. This is the program that you will use to edit your
plots, however, beyond that, it doesn't do much. That is why Pylux is also 
bundled with extensions to provide extra functionality. In the current 
version, Pylux comes bundled with the \texttt{texlux} and \texttt{plotter} 
extensions.

\subsection{\texttt{texlux}}
\texttt{texlux} is an extension to the base \texttt{editor} program which 
allows for the creation of reports in the \LaTeX{} format, which can then be 
post-processed to create a PDF file, or many other formats through the use 
of an external tool such as Pandoc.

\subsubsection{Commands}

\paragraph{\texttt{rn \textit{TEMPLATE OUTPUT TITLE}}}
Generates a report using the template \texttt{\textit{TEMPLATE}}, with the 
title \texttt{\textit{TITLE}}, writing the output to a file with path 
\texttt{\textit{OUTPUT}}.

\subsubsection{Processing}

\texttt{texlux} uses built-in functions to generate \LaTeX{} docments with 
pre-defined structures. Each built-in function has a corresponding \LaTeX{} 
style file installed in \texttt{\~{}/texmf} which is required to build 
the PDF report. Currently the only built-in function is \texttt{dimmer}, 
which produces a report categorised by dimmer and containing power draw 
totals.

\subsection{\texttt{plotter}}
\texttt{plotter} generates, from the fixture list and the fixture's symbol 
files, a plan view of the lighting plot in SVG format. It will consult the 
fixture's \texttt{posX} and \texttt{posY} tags to translate the symbol in 
the output image. It will also refer to the \texttt{rotation} tag to rotate 
the fixture symbol in the plot. Finally, it will colour the fixture using 
the hexadecimal colour code in the fixture's \texttt{colour} tag. If the 
\texttt{rotation} or \texttt{colour} tags are not present, \texttt{plotter} 
will calculate these when run.

There are also some options that can be set to customise the output of 
\texttt{plotter}.

\subsubsection{Commands}

\paragraph{\texttt{pn \textit{OUTPUT}}}
Generates a new SVG plot, writing the output to the file with path 
\texttt{\textit{OUTPUT}}.

\paragraph{\texttt{os \textit{OPTION VALUE}}}
Set the value of the option with name \texttt{\textit{OPTION}} to 
\texttt{\textit{VALUE}}.

\paragraph{\texttt{og \textit{OPTION}}}
Print the value of \texttt{\textit{OPTION}} to the console.

\paragraph{\texttt{oG}}
Print the values of all the options.

\subsubsection{Options}

\paragraph{\texttt{beam\_colour}}
The colour of the beam focus lines. Can be any colour in the legal list of 
gel colours or \texttt{auto} to make the focus lines the same colour as the 
fixture from which they come. Default: \texttt{Black}.

\paragraph{\texttt{beam\_width}}
The thickness of the beam focus lines in SVG points. Default: \texttt{6}.

\paragraph{\texttt{show\_beams}}
Choose whether or not to display the beam focus lines. Must be \texttt{True} 
or \texttt{False}. Default: \texttt{True}.

% -------- FILES SECTION ------- %
\section{Standard Tags} \label{sec:plotfile}
Below is a list of standard tags for each section, to advise which tags you 
should apply to your metadata and fixtures. Also included is a list of 
pseudo-tags: tags which are not added to the file but actually represent one 
or more other tags to make adding them easier.

\subsection{Standard Metadata Tags}
Whilst you can use any name for a tag you wish, there are some standard ones 
which are used by Pylux and its bundled extensions.

\paragraph{\texttt{production}}
The name of the production for which the lighting documentation is being 
produced, e.g. 'Romeo and Juliet'. Used by: \texttt{texlux}.

\paragraph{\texttt{designer}}
The name of the lighting designer for this production. Used by: 
\texttt{texlux}.

\paragraph{\texttt{board\_operator}}
The name of the person operating the main lighting board for this production.
Used by: \texttt{texlux}.

\paragraph{\texttt{spot\_operator}}
The name of the person operating the primary followspot for this production.
Used by: \texttt{texlux}.

\paragraph{\texttt{director}}
The name of the director of the production. Used by: \texttt{texlux}.

\paragraph{\texttt{venue}}
The location at which the production is taking place. Used by: 
\texttt{texlux}.

\subsection{Standard Fixture Data Tags} \label{sec:fixtags}

\paragraph{\texttt{dmx\_functions}}
This is the parent of a list of empty elements, each of which represents a 
function that the fixture has that requires the use of a DMX channel. For 
example, traditional fixtures will have an \texttt{intensity} function 
whilst modern LED fixtures may have \texttt{colour} and \texttt{mode} 
functions.

\paragraph{\texttt{dmx\_channels}}
The number of DMX channels that a fixture needs. This is automatically 
calculated from the \texttt{dmx\_functions} tag, so should not be changed 
manually.

\paragraph{\texttt{dmx\_start\_address}}
The start address of this fixture, if it has been addressed. This is 
automatically assigned using the address function so shouldn't be changed 
manually.

\paragraph{\texttt{universe}}
The universe in which the DMX channels for this fixture are located. This too 
is set when the address command is run so shouldn't be changed by the user. 

\paragraph{\texttt{posX}}
The x position in 2D space where this fixture is located. Measured in metres.

\paragraph{\texttt{posY}}
The y position in 2D space where this fixture is located. Measured in metres.

\paragraph{\texttt{focusX}}
The x position in 2D space where the centre of this fixture's beam is 
focused. Measured in metres.

\paragraph{\texttt{focusY}}
The y position in 2D space where the centre of this fixture's beam is 
focused. Measured in metres.

\paragraph{\texttt{rotation}}
The rotation of this fixture about its centre. Measured anticlockwise from 
the positive x axis in degrees. This can be automatically calculated if the 
preceding four data tags are present.

\paragraph{\texttt{circuit}}
For traditional fixtures only. The circuit into which the fixture is patched. 
Used by: \texttt{texlux}.

\paragraph{\texttt{power}}
For traditional fixtures only. The maximum power draw by the lamp in this 
fixture.

\paragraph{\texttt{dimmer\_uuid}}
For traditional fixtures only. The UUID of the dimmer (which must exist as a 
separate fixture in the plot file) which is controlling this fixture.

\paragraph{\texttt{dimmer\_chan}}
For traditional fixtures only. The name or number of the dimmer channel by 
which this fixture is controlled. 

\paragraph{\texttt{gel}}
The manufacturer's code of the gel that is being used in this fixture. The 
automatic colour calculation currently supports Rosco Supergel and E-colour 
and the named HTML (X11) colours.

\paragraph{\texttt{colour}}
A hexadecimal colour code indicating the colour which best represents the gel 
in this fixture. Can be automatically calculated if gel is present.

\subsubsection{Pseudo Fixture Tags}
These tags can be used to set multiple attributes of a fixture at once.

\paragraph{\texttt{position \textit{X},\textit{Y}}}
Sets \texttt{posX} to \texttt{\textit{X}} and \texttt{posY} to 
\texttt{\textit{Y}}.

\paragraph{\texttt{focus \textit{X},\textit{Y}}}
Sets \texttt{focusX} to \texttt{\textit{X}} and \texttt{focusY} to 
\texttt{\textit{Y}}.

\paragraph{\texttt{dimmer \textit{REF} \textit{CHAN}}}
Sets \texttt{dimmer\_uuid} to the uuid of the dimmer represented by 
\texttt{\textit{REF}} and \texttt{dimmer\_channel} to \texttt{\textit{CHAN}}.

\appendix
\section{Included Fixture Templates}
Here follows a list of the included fixture templates.
\begin{longtable}{|l|r|}
\hline
Fixture Type & Pylux Short Name \\ \hline
Generic Incandescent & generic \\
Hutton P650 & p650 \\
LEDJ RGB PAR & ledjrgbpar \\
PAR64 MFL & par64mfl \\
Strand Patt.23 & patt23 \\
Zero88 Betapack 2 & betapack2 \\
\hline
\end{longtable}

\section{Supported Gel Colours}
These are the gel names that can be assigned to a fixture and their colour be 
generated automatically.

\subsection{HTML Standard Colours}
Standard HTML/X11 named colours. Taken directly from the HTML colour 
definitions.
\begin{longtable}{|l|r|}
\hline
Pylux Gel Name & Rendered Colour \\ \hline
AliceBlue & \texttt{\#F0F8FF} \\
AntiqueWhite & \texttt{\#FAEBD7} \\
Aqua & \texttt{\#00FFFF} \\
Aquamarine & \texttt{\#7FFFD4} \\
Azure & \texttt{\#F0FFFF} \\
Beige & \texttt{\#F5F5DC} \\
Bisque & \texttt{\#FFE4C4} \\
Black & \texttt{\#000000} \\
BlanchedAlmond & \texttt{\#FFEBCD} \\
Blue & \texttt{\#0000FF} \\
BlueViolet & \texttt{\#8A2BE2} \\
Brown & \texttt{\#A52A2A} \\
BurlyWood & \texttt{\#DEB887} \\
CadetBlue & \texttt{\#5F9EA0} \\
Chartreuse & \texttt{\#7FFF00} \\
Chocolate & \texttt{\#D2691E} \\
Coral & \texttt{\#FF7F50} \\
CornflowerBlue & \texttt{\#6495ED} \\
Cornsilk & \texttt{\#FFF8DC} \\
Crimson & \texttt{\#DC143C} \\
Cyan & \texttt{\#00FFFF} \\
DarkBlue & \texttt{\#00008B} \\
DarkCyan & \texttt{\#008B8B} \\
DarkGoldenrod & \texttt{\#B8860B} \\
DarkGray & \texttt{\#A9A9A9} \\
DarkGreen & \texttt{\#006400} \\
DarkKhaki & \texttt{\#BDB76B} \\
DarkMagenta & \texttt{\#8B008B} \\
DarkOliveGreen & \texttt{\#556B2F} \\
DarkOrange & \texttt{\#FF8C00} \\
DarkOrchid & \texttt{\#9932CC} \\
DarkRed & \texttt{\#8B0000} \\
DarkSalmon & \texttt{\#E9967A} \\
DarkSeaGreen & \texttt{\#8FBC8F} \\
DarkSlateBlue & \texttt{\#483D8B} \\
DarkSlateGray & \texttt{\#2F4F4F} \\
DarkTurquoise & \texttt{\#00CED1} \\
DarkViolet & \texttt{\#9400D3} \\
DeepPink & \texttt{\#FF1493} \\
DeepSkyBlue & \texttt{\#00BFFF} \\
DimGray & \texttt{\#696969} \\
DodgerBlue & \texttt{\#1E90FF} \\
FireBrick & \texttt{\#B22222} \\
FloralWhite & \texttt{\#FFFAF0} \\
ForestGreen & \texttt{\#228B22} \\
Fuchsia & \texttt{\#FF00FF} \\
Gainsboro & \texttt{\#DCDCDC} \\
GhostWhite & \texttt{\#F8F8FF} \\
Gold & \texttt{\#FFD700} \\
Goldenrod & \texttt{\#DAA520} \\
Gray & \texttt{\#808080} \\
Green & \texttt{\#008000} \\
GreenYellow & \texttt{\#ADFF2F} \\
Honeydew & \texttt{\#F0FFF0} \\
HotPink & \texttt{\#FF69B4} \\
IndianRed & \texttt{\#CD5C5C} \\
Indigo & \texttt{\#4B0082} \\
Ivory & \texttt{\#FFFFF0} \\
Khaki & \texttt{\#F0E68C} \\
Lavender & \texttt{\#E6E6FA} \\
LavenderBlush & \texttt{\#FFF0F5} \\
LawnGreen & \texttt{\#7CFC00} \\
LemonChiffon & \texttt{\#FFFACD} \\
LightBlue & \texttt{\#ADD8E6} \\
LightCoral & \texttt{\#F08080} \\
LightCyan & \texttt{\#E0FFFF} \\
LightGoldenrodYellow & \texttt{\#FAFAD2} \\
LightGreen & \texttt{\#90EE90} \\
LightGrey & \texttt{\#D3D3D3} \\
LightPink & \texttt{\#FFB6C1} \\
LightSalmon & \texttt{\#FFA07A} \\
LightSeaGreen & \texttt{\#20B2AA} \\
LightSkyBlue & \texttt{\#87CEFA} \\
LightSlateGray & \texttt{\#778899} \\
LightSteelBlue & \texttt{\#B0C4DE} \\
LightYellow & \texttt{\#FFFFE0} \\
Lime & \texttt{\#00FF00} \\
LimeGreen & \texttt{\#32CD32} \\
Linen & \texttt{\#FAF0E6} \\
Magenta & \texttt{\#FF00FF} \\
Maroon & \texttt{\#800000} \\
MediumAquamarine & \texttt{\#66CDAA} \\
MediumBlue & \texttt{\#0000CD} \\
MediumOrchid & \texttt{\#BA55D3} \\
MediumPurple & \texttt{\#9370DB} \\
MediumSeaGreen & \texttt{\#3CB371} \\
MediumSlateBlue & \texttt{\#7B68EE} \\
MediumSpringGreen & \texttt{\#00FA9A} \\
MediumTurquoise & \texttt{\#48D1CC} \\
MediumVioletRed & \texttt{\#C71585} \\
MidnightBlue & \texttt{\#191970} \\
MintCream & \texttt{\#F5FFFA} \\
MistyRose & \texttt{\#FFE4E1} \\
Moccasin & \texttt{\#FFE4B5} \\
NavajoWhite & \texttt{\#FFDEAD} \\
Navy & \texttt{\#000080} \\
OldLace & \texttt{\#FDF5E6} \\
Olive & \texttt{\#808000} \\
OliveDrab & \texttt{\#6B8E23} \\
Orange & \texttt{\#FFA500} \\
OrangeRed & \texttt{\#FF4500} \\
Orchid & \texttt{\#DA70D6} \\
PaleGoldenrod & \texttt{\#EEE8AA} \\
PaleGreen & \texttt{\#98FB98} \\
PaleTurquoise & \texttt{\#AFEEEE} \\
PaleVioletRed & \texttt{\#DB7093} \\
PapayaWhip & \texttt{\#FFEFD5} \\
PeachPuff & \texttt{\#FFDAB9} \\
Peru & \texttt{\#CD853F} \\
Pink & \texttt{\#FFC0CB} \\
Plum & \texttt{\#DDA0DD} \\
PowderBlue & \texttt{\#B0E0E6} \\
Purple & \texttt{\#800080} \\
Red & \texttt{\#FF0000} \\
RosyBrown & \texttt{\#BC8F8F} \\
RoyalBlue & \texttt{\#4169E1} \\
SaddleBrown & \texttt{\#8B4513} \\
Salmon & \texttt{\#FA8072} \\
SandyBrown & \texttt{\#F4A460} \\
SeaGreen & \texttt{\#2E8B57} \\
Seashell & \texttt{\#FFF5EE} \\
Sienna & \texttt{\#A0522D} \\
Silver & \texttt{\#C0C0C0} \\
SkyBlue & \texttt{\#87CEEB} \\
SlateBlue & \texttt{\#6A5ACD} \\
SlateGray & \texttt{\#708090} \\
Snow & \texttt{\#FFFAFA} \\
SpringGreen & \texttt{\#00FF7F} \\
SteelBlue & \texttt{\#4682B4} \\
Tan & \texttt{\#D2B48C} \\
Teal & \texttt{\#008080} \\
Thistle & \texttt{\#D8BFD8} \\
Tomato & \texttt{\#FF6347} \\
Turquoise & \texttt{\#40E0D0} \\
Violet & \texttt{\#EE82EE} \\
Wheat & \texttt{\#F5DEB3} \\
White & \texttt{\#FFFFFF} \\
WhiteSmoke & \texttt{\#F5F5F5} \\
Yellow & \texttt{\#FFFF00} \\
YellowGreen & \texttt{\#9ACD32} \\
\hline
\end{longtable}

\subsection{Rosco E-colour and Supergel}
Colour representations taken from \url{http://stagedepot.co.uk}. Gels from the 
Supergel range are indicated by the preceding \# in the name.
\begin{longtable}{|l|r|}
\hline
Pylux Gel Name & Rendered Colour \\ \hline
\#24 & \texttt{\#F50014} \\
127 & \texttt{\#E16273} \\
166 & \texttt{\#F06165} \\
24 & \texttt{\#FF5A64} \\
157 & \texttt{\#FF92A3} \\
185 & \texttt{\#D5BFC1} \\
186 & \texttt{\#DCB9BD} \\
107 & \texttt{\#FFA5AF} \\
248 & \texttt{\#FFE2E4} \\
787 & \texttt{\#B9003C} \\
\#50 & \texttt{\#BB002C} \\
27 & \texttt{\#C8003C} \\
\#26 & \texttt{\#D70229} \\
46 & \texttt{\#DC005A} \\
29 & \texttt{\#E1003C} \\
26 & \texttt{\#E6003C} \\
\#46 & \texttt{\#EB016D} \\
106 & \texttt{\#F00032} \\
113 & \texttt{\#FF0064} \\
\#324 & \texttt{\#F50F39} \\
\#339 & \texttt{\#FF1283} \\
\#342 & \texttt{\#FF1562} \\
\#332 & \texttt{\#FF2957} \\
332 & \texttt{\#FF3787} \\
\#343 & \texttt{\#FF397F} \\
\#43 & \texttt{\#FF3E93} \\
148 & \texttt{\#FF507D} \\
748 & \texttt{\#E96785} \\
\#36 & \texttt{\#FF6D96} \\
\#336 & \texttt{\#FF73B7} \\
111 & \texttt{\#FF8CBE} \\
192 & \texttt{\#FF8CB4} \\
36 & \texttt{\#FFA0B9} \\
\#35 & \texttt{\#FFA7BB} \\
\#337 & \texttt{\#FFAFC2} \\
110 & \texttt{\#FFB4C8} \\
247 & \texttt{\#FAC3D7} \\
\#33 & \texttt{\#FFC2D0} \\
35 & \texttt{\#FFC8D2} \\
249 & \texttt{\#FFECF0} \\
49 & \texttt{\#BE0091} \\
126 & \texttt{\#BE009B} \\
\#39 & \texttt{\#E800BC} \\
\#344 & \texttt{\#FF05D3} \\
\#346 & \texttt{\#FF2DD5} \\
341 & \texttt{\#BC7DA8} \\
793 & \texttt{\#FF3CA0} \\
795 & \texttt{\#FA46C8} \\
128 & \texttt{\#FF50B4} \\
48 & \texttt{\#E66EAF} \\
328 & \texttt{\#FF64C8} \\
2 & \texttt{\#FF78DC} \\
170 & \texttt{\#E6AADC} \\
136 & \texttt{\#F0BEE6} \\
794 & \texttt{\#FFAFDC} \\
\#38 & \texttt{\#FFBBE2} \\
39 & \texttt{\#FFBDE1} \\
169 & \texttt{\#FADCF0} \\
797 & \texttt{\#AF0096} \\
798 & \texttt{\#A000BE} \\
\#49 & \texttt{\#C900E6} \\
\#349 & \texttt{\#F000FF} \\
\#47 & \texttt{\#CC4EB9} \\
345 & \texttt{\#CD6ED7} \\
703 & \texttt{\#D282DC} \\
704 & \texttt{\#F0AAFA} \\
705 & \texttt{\#EFAAFA} \\
707 & \texttt{\#6400B4} \\
700 & \texttt{\#7D00CD} \\
343 & \texttt{\#8C00D2} \\
\#347 & \texttt{\#B101DD} \\
\#358 & \texttt{\#8E0AEA} \\
\#348 & \texttt{\#DA2DFF} \\
701 & \texttt{\#B45ADC} \\
58 & \texttt{\#B46EF0} \\
194 & \texttt{\#BE8CF0} \\
52 & \texttt{\#E6AAFA} \\
702 & \texttt{\#E6D2F0} \\
\#54 & \texttt{\#E6C7FF} \\
\#351 & \texttt{\#EFDCFF} \\
750 & \texttt{\#F9F1FE} \\
\#382 & \texttt{\#250070} \\
181 & \texttt{\#5000AA} \\
799 & \texttt{\#3C00B4} \\
\#385 & \texttt{\#4F02CF} \\
\#74 & \texttt{\#4200FF} \\
\#374 & \texttt{\#4200FF} \\
\#59 & \texttt{\#7200FF} \\
706 & \texttt{\#6E46C8} \\
\#377 & \texttt{\#7124FF} \\
\#357 & \texttt{\#8A2BFF} \\
\#56 & \texttt{\#8C2FFF} \\
\#58 & \texttt{\#933FFD} \\
\#359 & \texttt{\#683FFF} \\
180 & \texttt{\#A064E6} \\
\#57 & \texttt{\#B482FF} \\
\#356 & \texttt{\#C38DFF} \\
\#355 & \texttt{\#A590FF} \\
137 & \texttt{\#C8B4E6} \\
\#55 & \texttt{\#C0AAFD} \\
\#353 & \texttt{\#C4ADFF} \\
\#52 & \texttt{\#DDBFFF} \\
\#53 & \texttt{\#E4DCFF} \\
71 & \texttt{\#0000B4} \\
\#384 & \texttt{\#0500D0} \\
\#83 & \texttt{\#0228EC} \\
508 & \texttt{\#5A46C3} \\
344 & \texttt{\#6662AE} \\
\#79 & \texttt{\#1626FF} \\
\#82 & \texttt{\#4F34F8} \\
199 & \texttt{\#6464E6} \\
\#84 & \texttt{\#5767FF} \\
\#78 & \texttt{\#6F6FFF} \\
142 & \texttt{\#AAAAF0} \\
\#371 & \texttt{\#A3A8FF} \\
709 & \texttt{\#C8C8FA} \\
\#372 & \texttt{\#D9DCFF} \\
53 & \texttt{\#E6E6FA} \\
713 & \texttt{\#003CA0} \\
120 & \texttt{\#005FBE} \\
\#85 & \texttt{\#0049CE} \\
716 & \texttt{\#0064D2} \\
\#80 & \texttt{\#0048FF} \\
715 & \texttt{\#3C6EDC} \\
714 & \texttt{\#5A91D2} \\
198 & \texttt{\#6478C8} \\
723 & \texttt{\#5082DC} \\
\#68 & \texttt{\#447DFF} \\
\#368 & \texttt{\#448AFF} \\
\#81 & \texttt{\#486FFF} \\
224 & \texttt{\#94A2BD} \\
75 & \texttt{\#64A0F0} \\
\#361 & \texttt{\#669EFC} \\
197 & \texttt{\#82AAE6} \\
710 & \texttt{\#8CA0F0} \\
711 & \texttt{\#AABEDC} \\
200 & \texttt{\#91BEF5} \\
525 & \texttt{\#8EB4FA} \\
719 & \texttt{\#9BC3F0} \\
712 & \texttt{\#A0BEF0} \\
\#62 & \texttt{\#A1CEFF} \\
500 & \texttt{\#B4D2FF} \\
708 & \texttt{\#DCEBFA} \\
\#373 & \texttt{\#E0E9FD} \\
502 & \texttt{\#E1F0FF} \\
\#76 & \texttt{\#005773} \\
\#75 & \texttt{\#007AAC} \\
85 & \texttt{\#006EB9} \\
363 & \texttt{\#006EC3} \\
\#71 & \texttt{\#0096C7} \\
119 & \texttt{\#0078C8} \\
195 & \texttt{\#006ED2} \\
722 & \texttt{\#008CD2} \\
132 & \texttt{\#00A0DC} \\
721 & \texttt{\#0082E6} \\
\#69 & \texttt{\#00A3F7} \\
\#369 & \texttt{\#00C6FF} \\
\#65 & \texttt{\#00A9FF} \\
79 & \texttt{\#3C8CD2} \\
\#67 & \texttt{\#14A9FF} \\
\#366 & \texttt{\#29C0F9} \\
143 & \texttt{\#64BED2} \\
68 & \texttt{\#46B4F0} \\
352 & \texttt{\#5AC8E1} \\
\#367 & \texttt{\#44A5FF} \\
165 & \texttt{\#5AC8EB} \\
\#64 & \texttt{\#50AEFD} \\
\#72 & \texttt{\#55CCFF} \\
196 & \texttt{\#78D2E6} \\
\#70 & \texttt{\#6CE5FF} \\
161 & \texttt{\#7DD2F5} \\
\#66 & \texttt{\#94EAFF} \\
366 & \texttt{\#AAD2F0} \\
\#63 & \texttt{\#A4D3FF} \\
283 & \texttt{\#AFD9F5} \\
174 & \texttt{\#AFE1FA} \\
\#363 & \texttt{\#ABE9FF} \\
191 & \texttt{\#C9DDE4} \\
201 & \texttt{\#C3E1FA} \\
720 & \texttt{\#C3E1FB} \\
717 & \texttt{\#C3E1FB} \\
281 & \texttt{\#CDE6FA} \\
718 & \texttt{\#D5ECFA} \\
501 & \texttt{\#D7EBFA} \\
63 & \texttt{\#D2F5FF} \\
\#61 & \texttt{\#D3EAFF} \\
202 & \texttt{\#D7F0FF} \\
61 & \texttt{\#DCF5FF} \\
\#395 & \texttt{\#00726A} \\
\#392 & \texttt{\#009493} \\
\#95 & \texttt{\#009C91} \\
\#93 & \texttt{\#01A3A0} \\
729 & \texttt{\#00AFAA} \\
727 & \texttt{\#00A5B4} \\
\#73 & \texttt{\#00A4B8} \\
116 & \texttt{\#00C8B9} \\
172 & \texttt{\#00DCDC} \\
\#370 & \texttt{\#01CDDF} \\
183 & \texttt{\#00D7E3} \\
141 & \texttt{\#00D2E6} \\
115 & \texttt{\#00EBC8} \\
118 & \texttt{\#00E1EB} \\
354 & \texttt{\#00F0D7} \\
601 & \texttt{\#90A0A0} \\
144 & \texttt{\#5AE1E6} \\
600 & \texttt{\#98A8AA} \\
353 & \texttt{\#61E8E3} \\
724 & \texttt{\#69E1EB} \\
140 & \texttt{\#87F0EB} \\
117 & \texttt{\#B4FAF5} \\
725 & \texttt{\#BEF2F3} \\
203 & \texttt{\#EBFCFF} \\
\#393 & \texttt{\#007150} \\
327 & \texttt{\#008C50} \\
\#94 & \texttt{\#00985D} \\
325 & \texttt{\#019879} \\
735 & \texttt{\#00BE78} \\
124 & \texttt{\#00DC78} \\
323 & \texttt{\#00E1AA} \\
322 & \texttt{\#41F5BE} \\
131 & \texttt{\#64FAD2} \\
241 & \texttt{\#AADCC3} \\
\#91 & \texttt{\#005E2C} \\
219 & \texttt{\#9BDCAF} \\
190 & \texttt{\#C5E0D1} \\
728 & \texttt{\#C8EBD2} \\
504 & \texttt{\#D2F5DC} \\
730 & \texttt{\#DCFFE6} \\
\#90 & \texttt{\#007F06} \\
736 & \texttt{\#00AA00} \\
90 & \texttt{\#00BE00} \\
\#389 & \texttt{\#29F433} \\
89 & \texttt{\#5ADC5A} \\
\#89 & \texttt{\#51F655} \\
122 & \texttt{\#78FA6E} \\
242 & \texttt{\#B9E6B9} \\
139 & \texttt{\#4BC300} \\
\#86 & \texttt{\#89FA19} \\
243 & \texttt{\#CDF5AF} \\
731 & \texttt{\#E1FAD7} \\
213 & \texttt{\#E6FCDC} \\
740 & \texttt{\#5A6E00} \\
\#386 & \texttt{\#7BD300} \\
738 & \texttt{\#AAFF00} \\
\#388 & \texttt{\#D0F54E} \\
505 & \texttt{\#E3FF5A} \\
121 & \texttt{\#B4FF64} \\
88 & \texttt{\#DCFF64} \\
138 & \texttt{\#DCFFA0} \\
189 & \texttt{\#D9E6C8} \\
244 & \texttt{\#E6FABE} \\
733 & \texttt{\#EBF7CF} \\
245 & \texttt{\#F0FCD2} \\
246 & \texttt{\#F5FFE6} \\
642 & \texttt{\#968200} \\
643 & \texttt{\#B4A000} \\
104 & \texttt{\#FFDC00} \\
\#312 & \texttt{\#FFEA00} \\
\#10 & \texttt{\#FFF200} \\
767 & \texttt{\#FFE600} \\
10 & \texttt{\#FFFF00} \\
100 & \texttt{\#F5FF00} \\
101 & \texttt{\#FFF500} \\
550 & \texttt{\#E3CA3C} \\
514 & \texttt{\#F5FF5A} \\
\#96 & \texttt{\#F3FF6B} \\
513 & \texttt{\#F5FF7D} \\
7 & \texttt{\#FAFAD2} \\
\#07 & \texttt{\#FDFAD1} \\
212 & \texttt{\#FFFAD7} \\
\#06 & \texttt{\#FCFADB} \\
746 & \texttt{\#6E3C00} \\
741 & \texttt{\#826400} \\
653 & \texttt{\#965500} \\
777 & \texttt{\#F58200} \\
\#15 & \texttt{\#FECB00} \\
15 & \texttt{\#FFCD00} \\
179 & \texttt{\#FFBE00} \\
105 & \texttt{\#FFA000} \\
158 & \texttt{\#FF8700} \\
768 & \texttt{\#FFC600} \\
770 & \texttt{\#FFB400} \\
\#14 & \texttt{\#FCD419} \\
\#11 & \texttt{\#FFD21A} \\
650 & \texttt{\#C8A862} \\
742 & \texttt{\#E19B50} \\
208 & \texttt{\#E6A550} \\
20 & \texttt{\#FFBE55} \\
207 & \texttt{\#F0B46B} \\
102 & \texttt{\#FFDC5F} \\
\#313 & \texttt{\#FFE462} \\
286 & \texttt{\#FFB464} \\
232 & \texttt{\#EDBE83} \\
744 & \texttt{\#F8C882} \\
441 & \texttt{\#FAC084} \\
204 & \texttt{\#FAC387} \\
\#09 & \texttt{\#FFCB86} \\
628 & \texttt{\#FFC88C} \\
\#13 & \texttt{\#FFD88F} \\
285 & \texttt{\#FCCD94} \\
765 & \texttt{\#FFE591} \\
13 & \texttt{\#FCD89B} \\
442 & \texttt{\#FCDCAD} \\
188 & \texttt{\#EEDABF} \\
205 & \texttt{\#FCD9B1} \\
764 & \texttt{\#FCE6B4} \\
103 & \texttt{\#FCEACC} \\
763 & \texttt{\#FCF0D2} \\
206 & \texttt{\#FCEAD6} \\
443 & \texttt{\#FCEFDB} \\
444 & \texttt{\#FAF3E8} \\
159 & \texttt{\#FFFAEB} \\
511 & \texttt{\#963C00} \\
512 & \texttt{\#CC6100} \\
507 & \texttt{\#F85000} \\
781 & \texttt{\#FF5000} \\
19 & \texttt{\#FF4600} \\
164 & \texttt{\#FF3200} \\
135 & \texttt{\#FF5F00} \\
22 & \texttt{\#FF6900} \\
778 & \texttt{\#FF7600} \\
\#23 & \texttt{\#FF5A00} \\
780 & \texttt{\#FF6F00} \\
\#22 & \texttt{\#FF430A} \\
\#19 & \texttt{\#FF390B} \\
\#21 & \texttt{\#FF6613} \\
\#317 & \texttt{\#FF7418} \\
\#20 & \texttt{\#FF871C} \\
\#40 & \texttt{\#FF4F1F} \\
\#318 & \texttt{\#FF6F29} \\
21 & \texttt{\#FF8C32} \\
652 & \texttt{\#FF8246} \\
25 & \texttt{\#FF6E46} \\
\#303 & \texttt{\#FF8A4A} \\
17 & \texttt{\#E68C64} \\
287 & \texttt{\#FFA055} \\
\#30 & \texttt{\#FF7A59} \\
651 & \texttt{\#FF9B5F} \\
622 & \texttt{\#FFA064} \\
134 & \texttt{\#FAA873} \\
156 & \texttt{\#E1B48C} \\
624 & \texttt{\#FFAA73} \\
8 & \texttt{\#FC9B80} \\
626 & \texttt{\#FFB482} \\
147 & \texttt{\#FCB98C} \\
237 & \texttt{\#FCB292} \\
236 & \texttt{\#FAC795} \\
184 & \texttt{\#D3C5BC} \\
187 & \texttt{\#E0C0B3} \\
\#04 & \texttt{\#F9B09A} \\
604 & \texttt{\#FFC295} \\
747 & \texttt{\#F5B99F} \\
\#01 & \texttt{\#FBB39A} \\
\#03 & \texttt{\#FBBA9A} \\
776 & \texttt{\#FCBE9B} \\
791 & \texttt{\#F6B8A9} \\
238 & \texttt{\#F5B9AA} \\
775 & \texttt{\#FCC6A4} \\
\#304 & \texttt{\#FABCA9} \\
790 & \texttt{\#FFB4A5} \\
108 & \texttt{\#FCBEA9} \\
\#02 & \texttt{\#FFD1AC} \\
773 & \texttt{\#FCC5B2} \\
9 & \texttt{\#FCD7B3} \\
4 & \texttt{\#FFC8B4} \\
774 & \texttt{\#FCD1BE} \\
152 & \texttt{\#FFD2C1} \\
151 & \texttt{\#FFCDC3} \\
506 & \texttt{\#FFDACE} \\
162 & \texttt{\#FCDED8} \\
223 & \texttt{\#FFF3E8} \\
\#27 & \texttt{\#B00202} \\
789 & \texttt{\#AA3C32} \\
\#25 & \texttt{\#E51F00} \\
182 & \texttt{\#F50000} \\
\#32 & \texttt{\#FF413C} \\
193 & \texttt{\#F37873} \\
\#31 & \texttt{\#FF847F} \\
779 & \texttt{\#FC9885} \\
\#331 & \texttt{\#FF9D8D} \\
176 & \texttt{\#FFAAA0} \\
\#305 & \texttt{\#F5BAB8} \\
109 & \texttt{\#FFB2AF} \\
153 & \texttt{\#FFC8C8} \\
749 & \texttt{\#FCD3CD} \\
154 & \texttt{\#FFD5CF} \\
\#05 & \texttt{\#FFD7D3} \\
299 & \texttt{\#373536} \\
211 & \texttt{\#67686A} \\
210 & \texttt{\#8E8E8E} \\
225 & \texttt{\#9FA0A2} \\
602 & \texttt{\#A5A5AA} \\
\#398 & \texttt{\#B0B4B9} \\
209 & \texttt{\#BABABA} \\
270 & \texttt{\#BEBEBE} \\
275 & \texttt{\#BEBEBE} \\
603 & \texttt{\#C8CDCD} \\
298 & \texttt{\#CCCECD} \\
226 & \texttt{\#F5F2EB} \\
3 & \texttt{\#FAF0FA} \\
278 & \texttt{\#F3FFF3} \\
279 & \texttt{\#FFF3F7} \\
503 & \texttt{\#F4FAFF} \\
218 & \texttt{\#F5FFFF} \\
217 & \texttt{\#F6FEFF} \\
221 & \texttt{\#F6FEFF} \\
130 & \texttt{\#FFFFFF} \\
250 & \texttt{\#FFFFFF} \\
258 & \texttt{\#FFFFFF} \\
400 & \texttt{\#FFFFFF} \\
429 & \texttt{\#FFFFFF} \\
450 & \texttt{\#FFFFFF} \\
\#100 & \texttt{\#FFFFFF} \\
\#122 & \texttt{\#FFFFFF} \\
228 & \texttt{\#FFFFFF} \\
251 & \texttt{\#FFFFFF} \\
269 & \texttt{\#FFFFFF} \\
401 & \texttt{\#FFFFFF} \\
430 & \texttt{\#FFFFFF} \\
452 & \texttt{\#FFFFFF} \\
\#101 & \texttt{\#FFFFFF} \\
\#124 & \texttt{\#FFFFFF} \\
229 & \texttt{\#FFFFFF} \\
252 & \texttt{\#FFFFFF} \\
402 & \texttt{\#FFFFFF} \\
432 & \texttt{\#FFFFFF} \\
460 & \texttt{\#FFFFFF} \\
\#104 & \texttt{\#FFFFFF} \\
\#125 & \texttt{\#FFFFFF} \\
220 & \texttt{\#FFFFFF} \\
253 & \texttt{\#FFFFFF} \\
271 & \texttt{\#FFFFFF} \\
404 & \texttt{\#FFFFFF} \\
434 & \texttt{\#FFFFFF} \\
462 & \texttt{\#FFFFFF} \\
\#00 & \texttt{\#FFFFFF} \\
\#113 & \texttt{\#FFFFFF} \\
\#126 & \texttt{\#FFFFFF} \\
254 & \texttt{\#FFFFFF} \\
272 & \texttt{\#FFFFFF} \\
410 & \texttt{\#FFFFFF} \\
464 & \texttt{\#FFFFFF} \\
\#114 & \texttt{\#FFFFFF} \\
\#127 & \texttt{\#FFFFFF} \\
214 & \texttt{\#FFFFFF} \\
255 & \texttt{\#FFFFFF} \\
273 & \texttt{\#FFFFFF} \\
414 & \texttt{\#FFFFFF} \\
480 & \texttt{\#FFFFFF} \\
\#119 & \texttt{\#FFFFFF} \\
\#132 & \texttt{\#FFFFFF} \\
215 & \texttt{\#FFFFFF} \\
256 & \texttt{\#FFFFFF} \\
274 & \texttt{\#FFFFFF} \\
416 & \texttt{\#FFFFFF} \\
481 & \texttt{\#FFFFFF} \\
\#120 & \texttt{\#FFFFFF} \\
\#140 & \texttt{\#FFFFFF} \\
129 & \texttt{\#FFFFFF} \\
216 & \texttt{\#FFFFFF} \\
257 & \texttt{\#FFFFFF} \\
420 & \texttt{\#FFFFFF} \\
482 & \texttt{\#FFFFFF} \\
\#121 & \texttt{\#FFFFFF} \\
\#160 & \texttt{\#FFFFFF} \\
\hline
\end{longtable}

\end{document}
